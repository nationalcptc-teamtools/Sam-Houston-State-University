%%%%%%%%%%%%%%%%%%%%%%%%%%%%%%%%%%%%%%%%%%%%%%%%%%%%%%%
%          Penetration Test Report Template           %
%                     cyber@cfreg                     %
%             https://hack.newhaven.edu/              %
%                                                     %
%                    Contributors:                    %
% TJ Balon - https://github.com/balon                 %
% Samuel Zurowski - https://github.com/samuelzurowski %
% Charles Barone - https://github.com/CharlesBarone   %
%%%%%%%%%%%%%%%%%%%%%%%%%%%%%%%%%%%%%%%%%%%%%%%%%%%%%%%
%%%%%%%%%%%%%%%%%%%%%%%%%%%%%%%%%%%%%%%%%%%%%%%%%%%%%%%%%%%%%%
%   Customization Settings
%%%%%%%%%%%%%%%%%%%%%%%%%%%%%%%%%%%%%%%%%%%%%%%%%%%%%%%%%%%%%%

% Aliases
\newcommand{\faculty}{}
\newcommand{\client}{An Example Company}
\newcommand{\cptc}{AEC}
\newcommand{\teamname}{New Haven Hacking Inc.}
\newcommand{\logopath}{images/general/logo.png}
\newcommand{\institution}{University of New Haven}
\renewcommand{\doctitle}{Penetration Test Report}

% Custom Colors
% Used for moderate threat level text for better readability compared to default yellow
\definecolor{threatyellow}{RGB}{180,180,0}
% Used to set title page background color. Useful if logo background is not transparent.
% This makes it so there is no square with a different background color.
\definecolor{titlebackground}{HTML}{ffffff} % Default: ffffff

% Toggle Team Members list on title page
\newboolean{teamMembersEnabled} % Do not modify this line
% Set to true to enable the list, false to disable it.
\setboolean{teamMembersEnabled}{true}

% Team Members
% One team member per line.
% Example: \teammember{name}{email@example.tld}
\newcommand{\teammembers}{
\teammember{Samuel Zurowski}{szuro1@unh.newhaven.edu}
\teammember{Charles Barone}{cbaro4@unh.newhaven.edu}
}

% Institution logo on title page
\newboolean{institutionLogoEnabled} % Do not modify this line
% Set to true to enable the institution logo, false to disable it.
\setboolean{institutionLogoEnabled}{true}
\newcommand{\institutionlogo}{images/general/insitutionlogo.png}


%%%%%%%%%%%%%%%%%%%%%%%%%%%%%%%%%%%%%%%%%%%%%%%%%%%%%%%%%%%%%%
%   DO NOT MODIFY BELOW HERE
%%%%%%%%%%%%%%%%%%%%%%%%%%%%%%%%%%%%%%%%%%%%%%%%%%%%%%%%%%%%%%

% Redefined Commands
\renewcommand\theadfont{\bfseries}

% Custom Commands
\newcommand\todo[1]{\textcolor{red}{#1}}
\newcommand{\teammember}[2]{{#1} & \href{mailto:{#2}}{{#2}}  \\}
% Example: \vuln{Lack of MariaDB Authentication}{7.5}{5}{10}
\newcommand{\vuln}[4]{#1 & \vulncolor{#2}#2 & \vulncolor{#3}#3 & \vulncolor{#4}#4 \\ \hline}
\newcommand{\vulncolor}[1]{
        \ifboolexpr{
            test {\ifdimless{#1 pt}{4.0pt}}
        }
        {\cellcolor{green}} % Score < 4.0 (Low)
        {\ifboolexpr{
            test {\ifdimless{#1 pt}{7.0pt}}
        }
        {\cellcolor{yellow}} % Score < 7.0 (Moderate)
        {\ifboolexpr{
            test {\ifdimless{#1 pt}{9.0pt}}
        }
        {\cellcolor{orange}} % Score < 9.0 (High)
        {\cellcolor{red}}}} % Score >= 9.0 (Critical)
}
\newcommand{\vulntext}[1]{\textbf{Threat Level}:\vulntextcolor{#1}}
\newcommand{\vulntextcolor}[1]{
        \ifboolexpr{
            test {\ifdimless{#1 pt}{4.0pt}}
        }
        {\color{green}{Low (#1)}} % Score < 4.0 (Low)
        {\ifboolexpr{
            test {\ifdimless{#1 pt}{7.0pt}}
        }
        {\color{threatyellow}{Moderate (#1)}} % Score < 7.0 (Moderate)
        {\ifboolexpr{
            test {\ifdimless{#1 pt}{9.0pt}}
        }
        {\color{orange}{High (#1)}} % Score < 9.0 (High)
        {\color{red}{Critical (#1)}}}} % Score >= 9.0 (Critical)
}

% Custom Environments
\newenvironment{vulntable}
    {
        \noindent
        The following table breaks down the discovered vulnerabilities by overall risk score, impact, and exploitability.  The scores were calculated using NIST's CVSS v3.1 calculator \cite{nvdcvss}.
        \begin{center}
            \textbf{Summary of Vulnerabilities by Base Score}
        \end{center}
        \begin{table}[h]
        \renewcommand{\arraystretch}{1.25}
        \centering
        \begin{tabular}{|>{\centering}p{15em}|c|>{\centering}p{4em}|c|}\hline
        \textbf{Risk Summary} & \textbf{Overall Risk Score} & \textbf{Impact} & \textbf{Exploitability} \\\hline
    }
    { 
        \end{tabular}
        \end{table}
    }

    
